\documentclass[12pt,french]{article}

\usepackage[T1]{fontenc}
\usepackage[utf8]{inputenc}
\usepackage{listings}
\usepackage{geometry}
\geometry{verbose,lmargin=2.5cm}
\usepackage{calc}
\usepackage{amssymb}
\usepackage{color}
\usepackage{amsfonts}
\usepackage{amsmath}
\pagestyle{empty}
\usepackage{babel}
\addto\extrasfrench{%
   \providecommand{\og}{\leavevmode\flqq~}%
   \providecommand{\fg}{\ifdim\lastskip>\z@\unskip\fi~\frqq}%
}

\title{\color{blue}Projet CPA \\ Vectorisation de boucle}

\author{Pierre Aubert, Salim Nahi, Puchen Liu}

% Pour compilé
% mkdir build
% cd build
% pdflatex ../rapportProjetCPA.tex

\begin{document}

\maketitle

\section*{\color{blue}Introduction}

Le but de ce projet est de fournir un plugin GCC et une librairie permettant d'étudier les accès mémoire du programme et ainsi déterminer si certaines boucles sont vectorisables ou non.

\section{\color{blue}Analyse du problème et architecture du projet}

Comme les plugins GCC 4.9 s'écrivent en C++, nous avons choisi le langage C++. Nous avons séparé le projet en deux grandes parties. La première est un plugin GCC qui permet d'ajouter des appels de fonction pour la deuxième partie qui elle est une librairies qui analyse les accès mémoires. Pour permettre une avancée simultannée des deux parties nous avons ajouté des fonctions en C qui permettent d'appeler la librairie manuellement et ce sans utiliser le plugin, et permettait aussi de tester le plugin sans avoir à utiliser la librairie. Nous avons également dévoloppé plusieurs tests afin de vérifier le bon fonctionnement des parties développées.

\section{\color{blue}L'analyse dynamique}

L'analyse dynamique conciste à rajouter des appels de fonction à la librairie d'analyse lors de la compilation. Ces fonctions seront appelées durant l'exécution du programme à analyser. 


\section{\color{blue}Le plugin}

Le plugin permet, dans un premier temps, d'insérer les appels à la librarie d'analyse lors de la compilation puis, dans un second temps d'effectuer une analyse statique des boucles les plus internes afin de déterminer avant l'exécution si elles sont vectorisables, ou non.

	\subsection{\color{blue}Spécification d'une fonction à analyser}

Pour spécifier une fonction à prendre en compte avec le plugin, il faut ajouter un pragma de la forme suivante :

% \inputencoding{latin9}
\begin{center}
\color{green}
\begin{lstlisting}
#pragma mihp vcheck ...
\end{lstlisting}
\end{center}
% \inputencoding{utf8}

Il est possible de préciser une fonction à analyser :

\begin{center}
\color{green}
\begin{lstlisting}
#pragma mihp vcheck fonction
\end{lstlisting}
\end{center}

Ou une liste de fonction à analyser :

\begin{center}
\color{green}
\begin{lstlisting}
#pragma mihp vcheck (func1, func2, func3, ..., funcN)
\end{lstlisting}
\end{center}

	\subsection{\color{blue}Analyse d'une boucle}

Dans ce projet, nous ne traiterons que les boucles les plus internes. Pour ce faire, nous analyserons chaque fonction spécifiée dans le \textbf{pragma mihp vcheck}. L'analyse conciste dans un premier temps de vérifier que la fonction spécifiée contient bien une boucle, une fois que la boucle la plus interne à été déterminée, le plugin ajoute différents appels à la librairie :
\vspace{0.5cm}
\begin{itemize}
\item Une fonction qui spécifie le début d'une boucle
\item Une fonction qui spécifie la fin d'une boucle
\item Une fonction qui spécifie une nouvelle iteration
\item Une fonction qui spécifie un accès mémoire
\end{itemize}
\vspace{0.5cm}

Le début de la boucle est précisé par la fonction
\textbf{mihp\_newLoop}
, cette fonction sera appelée juste avant le \textbf{header} de la boucle. La fonction \textbf{mihp\_endLoop} spécifie la fin d'une boucle et sera appelée sur chaque arrête sortante de la boucle. La fonction \textbf{mihp\_newIteration} indique une nouvelle itération, et sera appelée juste avant le
\textbf{latch} de la boucle. La fonction \textbf{mihp\_address} quant à elle, sera appelée pour chaque opérande d'un \textbf{statement gimple} du type \textbf{GIMPLE\_ASSIGN}.

Dans un premier temps nous avons pris en compte toutes les opérandes non constantes, puisque les opérandes constantes ne peuvent être que lues (ce qui ne pose aucun problème pour la vectorisation). Comme gimple converti toutes les expressions en diadiques ou en triadique, il créé des variables temporaires de la forme \textbf{D.XXXX}, les adresses de ces variables sont généralement \textbf{0x1}, \textbf{0x2}, \textbf{0x3}, ect. Nous avons donc fait des tests pour vérifier que cela ne posait pas de problème avec notre librairie d'analyse (qui sera détaillée dans la section suivante).

Nous avons vite mis en évidence que les variables temporaires créées par Gimple allaient nous poser problème, car elles ont toujours les mêmes adresses. La méthode naïve, consiste à tenir uniquement compte des varaibles décrites par \textbf{PARM\_DECL}, ce qui nous permet de n'utiliser que les variables définies dans le fichier source. Le problème c'est que ces varaibles sont toujours accédées en lecture. Il n'est donc pas possible de se contenter uniquement de ces varaibles.

Si nous prenons une itération simple, comme celle ci-dessous :

\begin{center}
\color{blue}
\begin{lstlisting}
c[i] = a[i] + b[i];
\end{lstlisting}
\end{center}

Il est naturel d'imaginer les appels de fonctions suivants :

\begin{center}
\color{blue}
\begin{lstlisting}
mihp_address(&b[i], sizeof(b[i]), MIHP_READ);
mihp_address(&a[i], sizeof(a[i]), MIHP_READ);
mihp_address(&c[i], sizeof(c[i]), MIHP_WRITE);
\end{lstlisting}
\end{center}

Mais si nous observons, ce que génère Gimple :

\begin{center}
\color{blue}
\begin{lstlisting}
D.2741 = (long unsigned int) i;
D.2742 = D.2741 * 4;
D.2743 = c + D.2742;              => &c[i]
D.2744 = (long unsigned int) i;
D.2745 = D.2744 * 4;
D.2746 = a + D.2745;              => &a[i]
D.2747 = *D.2746;                 => a[i]
D.2748 = (long unsigned int) i;
D.2749 = D.2748 * 4;
D.2750 = b + D.2749;              => &b[i]
D.2751 = *D.2750;                 => b[i]
D.2752 = D.2747 + D.2751;         => tmp = a[i] + b[i]
*D.2743 = D.2752;                 => c[i] = tmp
\end{lstlisting}
\end{center}

Donc, on vois dans cet exemple que si une varaible \textbf{PARM\_DECL} n'est pas l'opérande la plus a gauche, alors on lit d'adresse qui est dans l'opérande de gauche. Et si l'opérande de gauche est un \textbf{ADDR\_EXPR}, cette adresse sera alors écrite.


\section{\color{blue}La librairie d'analyse}

La librarie d'analyse effectue l'analyse des accès mémoire du programme et va indiquer si la boucle est vectorisable ou non, elle prend en entrée les differentes adresses mémoire des variables manipulées durant chaque itération de la boucle ainsi que le type de manipulation (lecture ou écriture) et compare dans un premier temps les adresses accédées entre une itération et ses suivantes ceci pour verifier si il y a des recouvrements entre les adresses mémoire (voir partie 4.2). Si recouvrement il y a, on teste le types de recouvrement en verifiant les types d'accés qui ont été efféctués sur ses adresses (voir tableau XX).

Dans l'implémentation de notre librarie nous avons définie 3 classe: Mihp\_Adress, Mihp\_Iteration, Mihp\_Loop pour recevoir les données envoyées grace au plugin, et pour permettre de faire l'analyse, voici ci-dessous un diagramme de nos classes: 	

\section{\color{blue}Les tests}

Pour tester le projet dans son ensemble et de manière efficace, nous avons choisi de permettre l'utilisation de la librairie d'analyse sans utiliser le plugin, ce qui nous a permis de débugger les deux parties de manière indémendantes. Nous avons donc réaliser des tests qui n'utilisent pas le plugin comme :

\begin{itemize}
\item TEST\_WITHOUT\_PLUGIN
\item TEST\_WITH\_NO\_PLUGIN
\item TEST\_WITHOUT\_PLUGIN\_VECTOR\_SIZE
\end{itemize}

Ces tests nous on permis de vérifier trois cas différents :

\begin{itemize}
\item La vectorisation est complétement impossible
\item La vectorisation est complétement possible
\item La vectorisation est possible mais avec une taille de vecteur limité
\end{itemize}

En ce qui concerne le plugin, nous avons testé, les analyses de fonctions seules et multiples, le cas où il y a plusieurs boucles dans une fonction, le cas où il y a un nid de boucle.


\section{\color{blue}Problèmes rencontrés}

Les principaux problèmes rencontrés lors de l'impémentation du plugin ont été pour ajouter l'appel à la fonction \textbf{mihp\_address}, car l'utilisation de la fonction \textbf{print\_gimple\_stmt} ne permet pas de vérifier dans tout les cas si le prototype de la fonction qui sera appelée par gimple est le bon ou non.

\section{\color{blue}Conclusion}



\end{document}


